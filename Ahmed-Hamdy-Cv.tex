%%%%%%%%%%%%%%%%%%%%%%%%%%%%%%%%%%%%%%%%%%%%%%%%%%%%%%%%%%%%%%%%%%%%%%
% LaTeX Template: Curriculum Vitae
%
% Source: http://www.howtotex.com/
% Feel free to distribute this template, but please keep the
% referal to HowToTeX.com.
% Date: July 2011
% 
%%%%%%%%%%%%%%%%%%%%%%%%%%%%%%%%%%%%%%%%%%%%%%%%%%%%%%%%%%%%%%%%%%%%%%
% This template is obtained from overleaf.com and edited by me (shakram02)
%%%%%%%%%%%%%%%%%%%%%%%%%%%%%%%%%%%%%%%%%%%%%%%%%%%%%%%%%%%%%%%%%%%%%%
\documentclass[paper=a4,fontsize=11pt]{scrartcl} % KOMA-article class
							
\usepackage[english]{babel}
\usepackage[utf8x]{inputenc}
\usepackage[protrusion=true,expansion=true]{microtype}
\usepackage{amsmath,amsfonts,amsthm}     % Math packages
\usepackage{graphicx}                    % Enable pdflatex
\usepackage[svgnames]{xcolor}            % Colors by their 'svgnames'
\usepackage{geometry}
	\textheight=700px                    % Saving trees ;-)
\usepackage{url}

\definecolor{AccentColor}{HTML}{15326A}
\definecolor{LinkColor}{HTML}{0000EE}
\usepackage{mdwlist}
\usepackage{hyperref}
\hypersetup{
    colorlinks=true,
    filecolor=magenta,      
    urlcolor=LinkColor
}
\urlstyle{same}

\frenchspacing              % Better looking spacings after periods
\pagestyle{empty}           % No pagenumbers/headers/footers
\geometry{margin=0.1in}		% Small margin
\setcounter{secnumdepth}{0}	% Disable section numbers
\usepackage{titlesec}
\titlespacing*{\section}
{0pt}{0.0ex plus 0.0ex minus 0ex}{0ex plus 0ex}
\titlespacing*{\subsection}
{0pt}{0.0ex plus 0.0ex minus 0ex}{0ex plus 0ex}
\titlespacing*{\subsubsection}
{0pt}{0.0ex plus 0.0ex minus 0ex}{0ex plus 0ex}
%%% Custom sectioning (sectsty package)
%%% ------------------------------------------------------------
% \usepackage{sectsty}

% \sectionfont{%			            % Change font of \section command
	% \usefont{OT1}{phv}{b}{n}%		% bch-b-n: CharterBT-Bold font
	% \sectionrule{0pt}{0pt}{-5pt}{3pt}}

%%% Macros
%%% ------------------------------------------------------------
\newlength{\spacebox}
\settowidth{\spacebox}{8888888888}			% Box to align text
\newcommand{\sepspace}{\vspace{1pt}}		% Vertical space macro

\newcommand{\MyName}[1]{ % Name
		\LARGE \usefont{OT1}{phv}{b}{n} #1
		 \normalsize \normalfont}
		
\newcommand{\MySlogan}[1]{ % Slogan (optional)
		\large \usefont{OT1}{phv}{m}{n} \textit{#1}
		\normalsize \normalfont}

\newcommand{\NewPart}[1]{\subsubsection{\color{AccentColor}{\textsc{#1}}}}
\newcommand{\Colored}[1]{\color{AccentColor}{#1}}
\newcommand{\BoldIt}[1]{\textbf{\textit{#1}}}
\newcommand{\colorsubsec}[1]{\subsection{\color{AccentColor}{#1}}}
\begin{document}

\MyName{Ahmed Hamdy Mahmoud}

\large{\textbf{Software Engineer}}

Mail \BoldIt{ahmedhamdyau@gmail.com} | Phone \BoldIt{+201148424331}

\href{https://www.github.com/shakram02/}{GitHub @/shakram02} |
\href{https://www.hackster.io/shakram02}{Hackster @/shakram02} |
\href{https://stackoverflow.com/users/4422856/shakram02}{stackoverflow @/shakram02}

\section{Experience}

\colorsubsec{Graphmented, BADRIT (Now Incorta Inc.), Alexandria - Software Engineer | November till 31st, December, 2018}
\begin{itemize*}
	\item One of best companies in Alexandria. The engineers there love their job and I enjoyed a good community
	\item Graphmented is an application to visualize data using AR Charts on the iOS platform
	\item My role was to design and develop features in the application and keeping in mind OOD and SOLID principles
	\item I learnt how to use a Mac PC up to using Swift with ARKit and writing code for the project in 8 working days
	\item After the company was acquired in December, I favored pursuing a masters and going back to research
\end{itemize*}

\colorsubsec{Repoxy: Replication Proxy for Trustworthy SDN Controller Operation [Mohamed Azab, PhD - Ahmed Hamdy, Undergrad - Ahmed Mansour, MSc] TrustCom 2018 | \href{https://github.com/shakram02/Repoxy}{Github} | \href{https://ieeexplore.ieee.org/document/8455887/}{IEEEXplore}}

\begin{itemize*}
	\item This paper was oriented towards controller security in SDN by applying controller replication
	\item The main contribution was enabling more than one controller to obtain the network state without knowing about the existence of each other, nor did the network know about the existence of the two controllers
	\item Implementation required careful reading of the OpenFlow protocol and a fair amount of packet manipulation and crafting. I also implemented some Ansible like software to automate testing across VMs
\end{itemize*}

\colorsubsec{PI Floor: Portable Interactive Floor with High Resilience and Minimal Setup for Edutainment [Ahmed Hamdy, Grad - Yara Abdullatif, Undergrad - Shaimaa Lazem, PhD] MUM 2018 | \href{https://github.com/shakram02/PiFloor}{Github} |
	\href{https://dl.acm.org/citation.cfm?doid=3282894.3289734}{ACM DL}}
\begin{itemize*}
	\item The main contribution of this poster was enabling education in areas with little resources through making an Android app that used paper tiles to represent an interactive grid where students can solve MCQ questions as a game where they move on the tiles to select the answer
	\item This project won the 1st place in ACM CAIRO CHI chapter's competition
	\item The implementation included an embedded WebSockets and HTTP server, using Android vision APIs and a VueJS app.
\end{itemize*}

\colorsubsec{Shixy, TMentors, Cairo - Intern Software Engineer | October 2015 to August 2016}
\begin{itemize*}
	\item Shixy is an advertisement robot that's operated through another endpoint in the network based on a Windows mini PC and Arduino
	\item My role was to re-design and implement the solutions (C\#/WPF) on the pilot's PC and on the robot. 
	\item The pilot's solution had a hardware joystick which I eventually replaced by a software one commands are sent over network to the robot's PC. The solution on the robot communicates with the Arduino which is responsible for the robot motion. It also featured image capturing and display
	\item Through the internship I self learned design patterns, SOLID, WPF and Dependency Injection and used them in the implementation.
\end{itemize*}
\subsubsection{Project: SIC/XE Assembler (Team of 3) \href{https://github.com/oddcoder/sickassembler}{GitHub}}
\begin{itemize*}
	\item Self learnt Rust language. Rust is a modern System development language by Mozilla. \href{https://www.rust-lang.org/}{rust-lang}
	\item Rust is a mixed paradigm language that offers C++ power and automatic memory management at compile time through the Ownership model, which gives it an initial  learning curve
	\item We followed code review system for each Pull Request and used Travis for automating our code testing and did extensive error checking for the input code files
	
\end{itemize*}
\colorsubsec{Education}
\begin{itemize*}
	\item Computer and Communications department, Faculty of Engineering, Alexandria University [2018] GPA: 3.09
	\item TOEFL(IBT) \BoldIt{106 [Oct, 2018]}
\end{itemize*}
\colorsubsec{Languages and Technologies}
\begin{itemize*}
	\item C/C++, Python, Arduino, Kotlin, Java, Rust, Android, Machine Learning, Git, Linux and NodeJs.
\end{itemize*}
\colorsubsec{Extracurricular Activities}
\subsubsection{Microsoft Student Partner | 2015 - 2016}
\begin{itemize*}
	\item Organizing on campus events and delivering the academic email to students
	\item Attending trainings and a hackathon with Microsoft employees in Cairo and other MSPs all over Egypt
\end{itemize*}

\begin{itemize*}
	\item MUM18 Organizing Student Volunteer; working with other international organizing committee and other SVs
	\item AIESEC Exchange Participant; educational project in Pontianak, Indonesia
\end{itemize*}
\end{document}


% \\

% % \href{https://www.linkedin.com/in/shakram02}{LinkedIn @/shakram02} |
% % \href{https://www.behance.net/shakram02}{Behance @/shakram02}.
% \noindent
% I enjoy exploration and being in dynamic environments. Coding and reading are one of my daily habits.
% %%% Education
% %%% ------------------------------------------------------------ %
% \NewPart{Education}

% \EducationEntry{Bsc. Computer and Communications Engineering}{Sep 2013 - Jul 2018}{Alexandria University}{%
% }
% \EducationEntry{TOEFL iBT}{October, 2018}{Score:106}{%
% }

% %%% Experience
% %%% ------------------------------------------------------------
% \NewPart{Industrial Experience}{}
% %\WorkEntry{Researcher}{Jun 2017}{IoT \& Cyber security Lab in Faculty of Engineering [Independent, Unpaid]}{Security research for Software Defined Networks (SDN).}
% %\sepspace
% \WorkEntry{Software Engineer}{Until 31/12/2018}{BADRIT (Acquired by Incorta Inc.)}{Team member in Graphmented | Using ARKit and SceneKit to make augmented reailty graphs\\}

% \WorkEntry{Intern Software Engineer}{October 2015 - August 2016}{TMentors}{Designed and implemented a software system to control a tel-operated advertisement robot, using C\# and WPF technologies.}
% %\sepspace

% %%% Skills
% %%% ------------------------------------------------------------
% % \NewPart{Skills}{}

% % \SkillsEntry{ \textsc{C, Kotlin, Python, Rust, Android}}
% % \SkillsEntry{ \textsc{git}, \textsc{Linux}}
% % \SkillsEntry{ \textsc{Arduino, Robotics}}


% %%% Publications 
% \NewPart{Publications}{}
% \ResearchEntry{Repoxy: Replication Proxy for Trustworthy SDN Controller Operation (Aug, 2018)}{Mohamed Azab, PhD}{Paper}{
% 	The 17th IEEE International Conference On Trust, Security And Privacy In Computing And Communications (IEEE TrustCom-2018) \href{https://ieeexplore.ieee.org/document/8455887/}{view on IEEE Xplore}
% }
% \ResearchEntry{PI Floor: Portable Interactive Floor with High Resilience and Minimal Setup for Edutainment}{Shaimaa Lazem, PhD}{Poster}{
% 	The 17th International Conference on Mobile and Ubiquitous Multimedia (MUM 2018) \href{https://dl.acm.org/citation.cfm?doid=3282894.3289734}{view on ACM DL}
% 	% \href{https://ieeexplore.ieee.org/document/8455887/}{view}
% }
% % \ResearchEntry{Security and Resilience for Network Traffic through Nature-Inspired Approaches - NCSR 2018}{}{}
% %%% Projects
% %%% ------------------------------------------------------------
% \NewPart{Familiar Technologies \& Fields}{}
% C, Java, Android, Rust, Embedded Systems (MCS51), Arduino, RPi, Linux (user and a bit about its internals)
% \NewPart{Sample Projects}{}
% \begin{itemize}
% 	%\item \href{https://github.com/shakram02/Arslanino}{Arslanino} [Kotlin/Firmata/Sockets] Transfer Arduino input signals to Arduino over network.
% 	\item \href{https://github.com/shakram02/Repoxy}{Repoxy} [OpenFlow, Low level socket I/O, OF Packet Fabrication] The first Software Defined Network (SDN) proxy to allow multiple controllers to manage OpenFlow switches
% 	\item \href{https://github.com/shakram02/PiFloor}{Pi Floor} [Android, Client/Server, Web Sockets] Interactive floor to enable educational games in classrooms
% 	      % \item \href{https://github.com/shakram02/Lwa-Web}{Lwa} [Android , NodeJs \& Arduino] Smart home lock using OneTimePasswords (OTP)
% 	\item \href{https://github.com/shakram02/Npuzzle-GUI}{N Puzzle} [AI Course | Kotlin] (College Project) N-Puzzle solver with a  GUI. Implements A*, BFS and DFS
% 	\item \href{https://github.com/shakram02/Lwa-Web} {Lwa} [Android, NodeJs, Arduino] A Clone of Amazon lock. That combines mobile, web and Arduino platforms
% 	\item Arduino Controller [Arduino, Java, Client/Server] (Proprietary) This application Reads and Updates states of multiple Arduino devices in the same LAN to be used in an industrial setting

% 	      %\item \href{https://github.com/shakram02/Prola-Client}{Prola} [Android / Kotlin] Don't buy barcode scanning hardware in shops, use Prola. Convert phones to barcode scanners, the server runs on the main PC to perform input.
% 	      %\item \href{https://github.com/alexsb-software/events-app}{Availability App} [Angular2, UI Design] Working with software committee members, this software facilitates task distribution in IEEE AlexSB when managing student events. \href{https://www.youtube.com/watch?v=ewxyfzxKQ8g}{Demo}.
% \end{itemize}

% %%% Projects
% %%% ------------------------------------------------------------
% \NewPart{Extracurricular Activities}{}

% \begin{itemize}
% 	\item AIESEC Exchange Participant, Educating high school students (English and Culture) in Pontianak, Indonesia
% 	      %	\item Researcher (student volunteer) in IoT and Cybersecurity lab, Faculty of Engineering, Alexandria University
% 	\item Student volunteer at MUM18 @ GUC Cairo, working with mostly German organizing committee and volunteers
% 	\item Microsoft Student Partner 2015 – 2016
% 	\item Software Committee member and course instructor (C, Arduino), IEEE Alexandria student branch
% 	\item Computer Science and Robotics educator for mid/high school students at \href{https://www.facebook.com/ProtonsAlexSB/}{Protons} for 3 times
% 	      % \item Made in Alex Robotics team member (competitions: IRC, Minesweepers)
% 	      %\item AIESEC member, Incoming global volunteer (IGV) committe; Project organization
% 	\item I blog about electronics for hobbyists on Hackster.io. Making sure to deliver byte sized concepts in details

% \end{itemize}
% % ----------- This is the end...
% \end{document}
